\documentclass[10pt,notitlepage]{article}
\usepackage{graphicx} 
\usepackage{verbatim} 
\usepackage[portuguese]{babel} 
\usepackage[utf8]{inputenc}
\usepackage[hmargin=2cm,vmargin=3.5cm,bmargin=2cm]{geometry}


\begin{document}

\subsubsection{Classe abstracta Event}

Classe com o conceito mais abstracto de Evento, contém as variáveis:
\begin{itemize}
\item \textit{String name};
\item \textit{String tipoActivity};
\item \textit{String location};
\item \textit{int maxParticipants};
\item \textit{int participants};
\item \textit{GregorianCalendar deadline};
\item \textit{GregorianCalendar date};
\item \textit{double duration};
\item \textit{TreeSet<User> participantsList};
\item \textit{TreeSet<Ranking> ranking};
\item \textit{TreeSet<Ranking> desistentes};
\item \textit{TreeSet<Simulacao> simula};
\end{itemize}
respectivos \textit{getters} e \textit{setters} e os vários construtores. Ainda tem métodos auxiliares para, adicionar um \textit{User}, \textit{Ranking} (desistente ou não) e \textit{Simulaçao} aos respectivos \textit{Sets} e para mostrar a classificação geral do evento.


\subsubsection{Tipo de Evento}

Subclasses de Evento (Marathon, HalfMarathon, MarathonBTT e Trail), todas estas contem mais uma variável \textit{distance}, que nos casos de Marathon e HalfMarathon são variáveis \textit{final}, porque este tipo de eventos tem distâncias especificas. Não tem métodos auxiliares para além de \textit{getDistance()}.


\subsubsection{Classe abstracta Activity}

Esta é a classe mais abstracta que contem o conceito de actividade. Contém variáveis comuns a todas as actividades:
\begin{itemize}
\item \textit{String name};
\item \textit{GregorianCalendar date};
\item \textit{double timeSpent};
\item \textit{double calories};
\end{itemize}
tal como os construtores, \textit{getters} e \textit{setters}.

\subsubsection{Classe abstracta Person}
Classe geral para todo tipo de utilizador. As suas variáveis são:
\begin{itemize}
\item \textit{String email};
\item \textit{String password};
\item \textit{String name};
\item \textit{char gender};
\item \textit{GregorianCalendar dateOfBirth};
\end{itemize}
e contém os métodos construtores \textit{getters} e \textit{setters}

\subsubsection{Classes User e Admin}

As subclasses de Person referem-se a dois possíveis tipos de utilizador; utilizador normal ou utilizador com privilégios de administrador.\\
A classe Admin não tem métodos ou variáveis adicionais, visto que este tipo de utilizador apenas opera sobre a base de dados da aplicação.\\
A classe User adiciona as seguintes variáveis:
\begin{itemize}
\item \textit{int height};
\item \textit{double weight};
\item \textit{String favoriteActivity};
\item \textit{TreeSet<Activity> userActivities};
\item \textit{TreeSet<String> friendsList};
\item \textit{TreeMap<String, ListRecords> records};
\item \textit{TreeSet<String> messageFriend};
\end{itemize}
Respectivos métodos \textit{getters} e \textit{setters}, construtores e métodos auxiliares para gerir os seus amigos, recordes, as suas actividades e estatísticas relevantes. Ainda contém funções auxiliares para a simulação de eventos.

\subsubsection{Comparators}
O tipo Person tem apenas um comparator:
\begin{itemize}
\item ComparePersonByName - que ordena por ordem alfabética do seu nome.
\end{itemize}


\subsubsection{Statistics}

A classe Statistics é usada para mostrar ao utilizador dados relevantes das suas actividades, estes podem ser descriminados por um dado mês ou por um ano. As suas variáveis são:
\begin{itemize}
\item \textit{double timeSpend};
\item \textit{double calories};
\item \textit{double distance};
\end{itemize}
contém os respectivos métodos \textit{getters} e \textit{setters} e construtores.


\end{document}














