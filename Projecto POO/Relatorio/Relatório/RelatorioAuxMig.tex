\documentclass[10pt,notitlepage]{article}
\usepackage{graphicx} 
\usepackage{verbatim} 
\usepackage[portuguese]{babel} 
\usepackage[utf8]{inputenc}
\usepackage[hmargin=2cm,vmargin=3.5cm,bmargin=2cm]{geometry}


\begin{document}

%%%CAPA%%%

\begin{figure}
\centering
\includegraphics[scale=0.5]{logo.pdf}
\end{figure}

$\\$
$\\$

\begin{center}


\subsection{Classe abstracta Event}

Classe com o conceito mais abstracto de Evento, contém as variáveis \textit{name}, \textit{tipoActivity}, \textit{location}, \textit{maxParticipants}, \textit{participants}, \textit{deadline}, \textit{date}, \textit{duration}, \textit{participantsList}, \textit{ranking}, \textit{desistentes} e \textit{simula}, respetivos \textit{getters} e \textit{setters} e os vários contrutores. Ainda tem métodos auxiliares para, adicionar um \textit{User}, \textit{Ranking} (desistente ou nao) e \textit{Simulaçao} aos respetivos \textit{Sets} e para mostrar a classificação geral do evento.


\subsubsection*{Tipo de Evento}

Subclasses de Evento (Marathon, HalfMarathon, MarathonBTT e Trail), todas estas contem mais uma variavel \textit{distance}, que nos casos de Marathon e HalfMarathon são variaves \textit{final}, porque este tipo de eventos tem distâncias especificas. Não tem métodos auxiliares para além de \textit{getDistance()}.










\end{document}

















